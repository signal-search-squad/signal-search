\documentclass[11pt,letterpaper]{article}
%\textwidth 6.25in \oddsidemargin 0.25in \evensidemargin 0.0in
\oddsidemargin .00in
\evensidemargin .00in
\topmargin 10 pt \headheight 0in \textheight 8.25in
\usepackage{amsmath, amsfonts, amsthm, amssymb}
\usepackage{units}
\usepackage{ulem}
\usepackage{ifpdf}
\usepackage{accents}
%\usepackage{setspace}
%\usepackage[utf8]{inputenc}
\usepackage[english]{babel}
\usepackage{graphicx}
\usepackage{float}
\usepackage{caption}
\usepackage{subcaption} 
\usepackage{cancel}
\usepackage{bm}
\usepackage[usenames,dvipsnames]{color}
\setlength{\parindent}{0pt}
\setlength{\parskip}{2ex}
\numberwithin{equation}{section}
\begin{document}

\title{Higher Dimensional Operators in EFT's}

\maketitle

An Effective Field Theory (EFT) has a finite number of parameters for given energy scale, $E$, and accuracy, $\epsilon$.
\begin{itemize}
\item For each set of interactions of some dimension $D=k-4$, there are a fininte number of parameters that describe the set of interactions.
\item The coefficient of terms with dimension $k-4$ are proportional to:
\[\left(\frac{E}{M}\right)^k\]
\item This means we only need to include terms up to dimension $k^*-4$ such that \[\left(\frac{E}{M}\right)^{k^*}\approx\epsilon\]
\end{itemize}

%%%MAKE FIGURES%%%%
%\begin{figure}[h!]
%	\includegraphics[width=\textwidth]{name_of_file}
%	\caption{}\label{}
%	\centering
%\end{figure}

%%MAKE TABLES%%
%\begin{table}{h!}
%\caption{}\label{}
%\begin{tabular}{}
%\end{tabular}
%\end{table}
\end{document}
